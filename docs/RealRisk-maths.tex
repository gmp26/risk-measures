\documentclass[12pt]{article}
\renewcommand{\arraystretch}{1.0}
\renewcommand{\baselinestretch}{1.0}
\setlength{\oddsidemargin}{-0.15in}
\setlength{\evensidemargin}{-0.15in} \setlength{\textwidth}{6.6in}
\setlength{\topmargin}{-0.2in} \setlength{\textheight}{9.1in}
\setlength{\fboxsep}{1cm}
% ------------------------------------------------------
\usepackage{graphicx}

\pagestyle{myheadings} \markright{RealRisk Mathematics}


\begin{document}
\parindent=0pt
\parskip=5pt

\section*{RealRisk Mathematics}

Let the baseline risk be $r$.  The risk in the 'active' group, $p$, depends on the measure of change
\begin{itemize}
\item {\bf Relative risk $RR$.}  By definition, $RR = p/r$.  So the final risk is $p = r \times RR$.
\item {\bf Percentage change  $PC$.}   The final risk is $r + r \times PC/100$.
\item {\bf Odds ratio  $OR$.}  By definition, $OR = \frac{p}{(1-p)} / \frac{r}{(1-r)}$.  Solving gives $p = 1- \frac{1}{(1+ OR(1-r)/r)}$.
\item {\bf Hazard ratio  $HR$.}   By definition, $HR = h_1(t)/h_0(t)$, where $h_1(t), h_0(t)$ are the  hazards in the 'active' and baseline groups respectively. Therefore $HR = H_1(t)/H_0(t)$, where $H_1(t), H_0(t)$ are the cumulative hazards. Now $H_1(t) = -\log S_1(t), H_0(t) = -\log S_0(t)$, where $S_1(t), S_0(t)$ are the survival probabilities.   And so $HR = \log S_1(t)/ \log S_0(t).$\\

For a specified follow-up time $t$, we have risks $p = 1- S_1(t)$, $r = 1- S_0(t)$, and so $HR = \log (1-p) / \log (1-r)$.

Rearranging gives  $p = 1 - (1-r)^{HR}$.
\end{itemize}






 \end{document}
